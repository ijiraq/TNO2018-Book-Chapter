\documentclass{aastex62}

\newcommand{\vdag}{(v)^\dagger}
\newcommand\aastex{AAS\TeX}
\newcommand\latex{La\TeX}

%% Tells LaTeX to search for image files in the 
%% current directory as well as in the figures/ folder.
\graphicspath{{./}{figures/}}


%% Reintroduced the \received and \accepted commands from AASTeX v5.2
\received{October 1, 2018}
\revised{---}
\accepted{---}
%% Command to document which AAS Journal the manuscript was submitted to.
%% Adds "Submitted to " the arguement.
\submitjournal{ApJ}

\shorttitle{Alternative Theories}
\shortauthors{Kavelaars et al.}

\watermark{NRC-DRAFT}

\begin{document}

\title{Perspectives on the distribution of orbits of distant Trans-Neptunian Objects}

\correspondingauthor{J. J. Kavelaars}
\email{JJKavelaars@gmail.com}
\author[0000-0001-7032-5255]{J. J. Kavelaars}
\affil{Department of Physics and Astronomy, University of Victoria, Elliott Building, 3800 Finnerty Rd, Victoria, BC V8P 5C2, Canada}
\affil{Herzberg Astronomy and Astrophysics Research Centre, National Research Council of Canada, 5071 West Saanich Rd, Victoria, British Columbia V9E 2E7, Canada}


\author[0000-0001-5368-386X]{Samantha M. Lawler}
\affiliation{NRC-Herzberg Astronomy and Astrophysics, National Research Council of Canada, 5071 West Saanich Rd, Victoria, British Columbia V9E 2E7, Canada}

\author[0000-0003-3257-4490]{Michele T. Bannister}
\affiliation{Astrophysics Research Centre, School of Mathematics and Physics, Queen's University Belfast, Belfast BT7 1NN, United Kingdom}

\author[0000-0002-3507-5964]{Cory Shankman}
\affiliation{Department of Physics and Astronomy, University of Victoria, Elliott Building, 3800 Finnerty Rd, Victoria, BC V8P 5C2, Canada}


\begin{abstract}

Hello, that was abstract. 

\end{abstract}

%% Keywords should appear after the \end{abstract} command. 
%% See the online documentation for the full list of available subject
%% keywords and the rules for their use.
\keywords{Kuiper belt, orbits --- biases ---  surveys}

%% From the front matter, we move on to the body of the paper.
%% Sections are demarcated by \section and \subsection, respectively.
%% Observe the use of the LaTeX \label
%% command after the \subsection to give a symbolic KEY to the
%% subsection for cross-referencing in a \ref command.
%% You can use LaTeX's \ref and \label commands to keep track of
%% cross-references to sections, equations, tables, and figures.
%% That way, if you change the order of any elements, LaTeX will
%% automatically renumber them.
%%
%% We recommend that authors also use the natbib \citep
%% and \citet commands to identify citations.  The citations are
%% tied to the reference list via symbolic KEYs. The KEY corresponds
%% to the KEY in the \bibitem in the reference list below. 

\section{Empty}


\section{Biases in detection of distant solar system objects.}

We perceive the reality of the universe through our observation of that universe and our understanding of those observations are biased by our observation and perception biases. This truism of observational science must be kept in mind when considering the review of our data. As a group we examine data and look for meaning in the correlations between quantities and strongly cling to our initial conceptions.  To see beyond our perception biases is difficult and perhaps intractable. 

The Kuiper belt is over 4.5 billion km distant from the Earth bound observer with the most distant objects known three times further away still.  The challenge of detecting objects at these great distances should not be under-estimated.  The Sun's light travels 

\section{Dynamical effects expected to be imprinted on the distant Kuiper belt by the presence of a massive planet}

\section{Diffusion and motion of large semi-major axes orbits}

\section{Destectibility of orbital effects}

\section{Biases in the angle of pericentre detection in the large-q large-a TNO sample}

\section{Conclusion: The orbits of Sedna and 2012 VP112 are weird.}


% \bibliographystyle{apj}
\bibliography{citations}

%% This command is needed to show the entire author+affilation list when
%% the collaboration and author truncation commands are used.  It has to
%% go at the end of the manuscript.
%\allauthors

%% Include this line if you are using the \added, \replaced, \deleted
%% commands to see a summary list of all changes at the end of the article.
%\listofchanges

\end{document}

% End of file `sample62.tex'.
